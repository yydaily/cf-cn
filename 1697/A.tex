\documentclass{article}
\usepackage{CJKutf8}
\usepackage[table]{xcolor}
\usepackage{array}
\newcolumntype{L}[1]{>{\raggedright\let\newline\\\arraybackslash\hspace{0pt}}m{#1}}
\begin{document}
\setlength{\parindent}{0pt}
\begin{CJK}{UTF8}{gbsn}
\begin{center}
\section*{A. 公园步行}
时间限制:1s\\
空间限制:256M\\
输入:标准输入\\
输出:标准输出
\end{center}
你在家附近的公园步行。公园有 $n+1$ 个长椅,从左到右编号分别从 $1$ 到 $n+1$。其中,第 $i$ 个长椅和第 $i+1$ 个长椅的距离是 $a_i$ 米。\\

初始状态,你有 $m$ 点力量。每走 $1$ 米,需要消耗 $1$ 点力量。当没有力量的时候,将不能行走。当然,你可以通过\textbf{坐在长椅上休息}来恢复力量(并且,这是唯一恢复力量点的方式)。当你坐着的时候,你能恢复任意数量的力量(坐的越久,恢复的越多)。但是你的力量\textbf{能超过} $m$。\\

你的任务时找到保证能从 $1$ 号长椅走到 $n+1$ 号长椅的\textbf{最小} 的能量\textbf{恢复量}。\\

一共包含 $t$ 组独立的测试用例。

\subsection*{输入}

第一行包含一个整数 $t(1\leq t \leq 100)$ —— 表示测试用例的组数。接下来是 $t$ 组样例。\\

每组用例的第一行包含两个整数 $n$ 和 $m(1\leq n \leq 100; 1\leq m \leq 10^4)$。\\

每组用例的第二行包含 $n$ 个整数 $a_1, a_2, \dots, a_n(1\leq a_i \leq 100)$,其中 $a_i$ 表示第 $i$ 个长椅到第 $i+1$ 个长椅的距离。

\subsection*{输出}

对于每组测试用例,输出一个整数,从第 $1$ 个椅子走到第 $n+1$ 个椅子所需要的 \textbf{最小的恢复量}(通过坐在椅子上)。

\subsection*{样例}

\begin{tabular}{|L{11.5cm}|}
\hline
输入\\
\hline
\rowcolor{gray!20} 3\\
\rowcolor{gray!20} 3 1\\
\rowcolor{gray!20} 1 2 1\\
\rowcolor{gray!20} 4 5\\
\rowcolor{gray!20} 3 3 5 2\\
\rowcolor{gray!20} 5 16\\
\rowcolor{gray!20} 1 2 3 4 5\\
\hline
输出\\
\hline
\rowcolor{gray!20} 3\\
\rowcolor{gray!20} 8\\
\rowcolor{gray!20} 0\\
\hline
\end{tabular}

\subsection*{说明}

在第一个测试用例中,你可以走到长椅 $2$,花费 $1$ 点力量,然后坐在第 $2$ 个长椅上恢复 $2$ 点力量,然后走到第 $3$ 个长椅,花费 $2$ 点力量,恢复 $1$ 点力量,然后走到第 $4$ 个长椅。\\

在第三个测试用例中,你有足够的力量直接走到第 $6$ 个长椅上,而不需要休息。

\newpage

\begin{center}
\section*{B. 促销}
时间限制:2s\\
空间限制:256M\\
输入:标准输入\\
输出:标准输出
\end{center}

商店一共售卖 $n$ 个商品,第 $i$ 个商品的售价是 $p_i$。商店的管理员准备进行一次促销:如果一个客户买了至少 $x$ 个商品,那么最便宜的 $y$ 个商品就免费。\\

管理员还没决定 $x$ 和 $y$ 的具体值。因此,他会询问你 $q$ 次:对于给定的 $x$ 和 $y$,如果一个客户进行\textbf{一次购买},能优惠的最大价格。\\

注意,所有的询问都是独立的;他们不会影响商店的库存。

\subsection*{输入}

第一行包含两个整数 $n$ 和 $q(1\leq n, q \leq 2\cdot 10^5)$ —— 表示商店中物品的数量和询问的次数。\\

第二行包含 $n$ 个整数 $p_1, p_2, \dots, p_n(1\leq p_i \leq 10^6)$,其中 $p_i$ —— 表示第 $i$ 个物品的价格。\\

接下来 $q$ 行,每行包含两个整数 $x_i$ 和 $y_i(1\leq y_i\leq x_i\leq n)$ —— 表示在第 $i$ 次询问中,参数 $x$ 和 参数 $y$。

\subsection*{输出}

对于每次询问,输出一个整数 —— 表示\textbf{一次购买中},最大的优惠价格。

\subsection*{样例}

\begin{tabular}{|L{11.5cm}|}
\hline
输入\\
\hline
\rowcolor{gray!20} 5 3\\
\rowcolor{gray!20} 5 3 1 5 2\\
\rowcolor{gray!20} 3 2\\
\rowcolor{gray!20} 1 1\\
\rowcolor{gray!20} 5 3\\
\hline
输出\\
\hline
\rowcolor{gray!20} 8\\
\rowcolor{gray!20} 5\\
\rowcolor{gray!20} 6\\
\hline
\end{tabular}

\subsection*{说明}

在第一次询问中,客户能买价值为 $5, 3, 5$ 的三个物品,其中两个最便宜的是 $3+5=8$。\\

在第二次询问中,客户能买两个价值为 $5$ 和 $5$ 的物品,其中最便宜的是 $5$。\\

在第三次询问中,客户买下了所有的物品,其中最便宜的三个是免费的;他们的总价值是 $1+2+3=6$。

\end{CJK}
\end{document}